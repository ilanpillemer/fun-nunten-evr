\documentclass{article}
%\usepackage[margin=2in]{geometry}
\usepackage[osf,p]{libertinus}
\usepackage{microtype}
\usepackage[pdfusetitle,hidelinks]{hyperref}

\usepackage[series={A,B,C}]{reledmac}
\usepackage{reledpar}

\usepackage{graphicx}
\usepackage{polyglossia}
\setmainlanguage{english}
\setotherlanguage{hebrew}
\gappto\captionshebrew{\renewcommand\chaptername{קאַפּיטל}}
\usepackage{metalogo}


%%linenumincrement*{1}
\firstlinenum*{15}
%%\setlength{\Lcolwidth}{0.44\textwidth}
%%\setlength{\Rcolwidth}{0.44\textwidth}

\begin{document}
%%\maxhnotesA{0.8\textheight}
\renewcommand{\abstractname}{\vspace{-\baselineskip}}
\title{Fun noentu over, Vol I, p172.}
\author{Transl. Ilan Pillemer}
\date{\today}

\maketitle



\begin{pairs}

\begin{Rightside}

\begin{RTL}
\begin{hebrew}
\beginnumbering
\autopar
\emph{
ייִדיש טעאַטער אין די „די־פּי“־לאַגערן פֿון עסטרײך (אַמעריקאַנד זאָנע)
}
\newline

אויך אין די עסטרײכישע ד“פּ־לאַגערן אין די יאָרן 1946־1947
געגרינדעט געװאָרן פּראָפֿעסיאָנעלע טעאַטער־גרופּן, װי אויך אַ צאָל דראַמאַטישע קרײזן.
 צוליב דער קלײנער צאָל געראַטעװעטע ײדין, װעלכע האָבן זיך געפֿונען אין עסטרײך, איז אויך די צאָל אָנטײלנעמער און
 דער פֿאַרנעם פֿון די טעאַטער־אונטערנעמונג געװען אַ באַשײדענער.

דער ערשטער טעאַטער־קאָלעקטיװ האָט זיך געשאַפֿן אין 1946 און האָט זיך גערופן „ע ס ט ר י י כ י ש ע ר\space\space  י י ד י ש ע ר\space\space  ט ע אַ ט ע ר “.
דער אַנסאַמבל איז אַדמיניסטראַטיװ און קינסטלעריש אָנגעפֿירט געװאָרן פֿון באַקאַנטן אַקטיאָר ח. קרעלמאן.
דער טעאַטער האָט געשפּילט דעם אַלטן יידישן טעאַטער־רעפּערטואַר און האָטב בײם  
 טעאַטער־עולם ארויסגערופֿן אָנערקענונג.


די ײדישע ד“פּ־לאַנגערן אין עסטרײך האָבן באַקומען די מעגלעכקײט צו זען אַזעלכע טעאַטער־פֿאָרשטעלונגען װי „דער דיבוק“ (אַנסקי),
„גאָט, מענטש און טײװל“ (גאָרדין), „יענקל בוילע“ (קאָברין), „כאַסיע די יתומה“ (גאָרדין),
„הערשעלע אָסטראָפּאָלער“ (גערשענזאָן).

דער קאָלעקטיװ האָט נישט געהאַט קײן ספּעציעלע קינסטלערישע אַמביציעס, נישט געזוכט קײן נײע װעגן און נײע פּיעסן;
נישט אָפּגעשפּיגלט די בלוטיקע פֿאַרגאַנגענהײט װאָס איז נאָך דאַן געװען זײער נאָענט.
זײ האָבן אָבער געגעבן גוטע און רײנע פֿאָרשטעלונגען און דאָס איז געװען זײער פּאַרדינסט.
דער אַנסאַמל האָט געהאַט זײן רעזידענץ אין לינץ און איז מאַטעריעל
און מאָראַליש אונטערגעשטיצט געװאָרן פֿון דזשאָינט און די
נײ־געשאַפֿענע ײדישע אָרגאַניזאַציעס.

\endnumbering
\end{hebrew}
\end{RTL}
\end{Rightside}


\begin{Leftside}
\begin{english}
\beginnumbering
\autopar
\emph{
Yiddish theatre in the D.P camps in Austria (American Zone)
}
\newline 
 
Professional theatre-groups, as well as a number of drama circles, were also founded in the Austrian D.P. camps in the years 1946-1947.
On account of the small number of Jewish survivors who found themselves in Austria - both the number of participants; and, the forming of
the theatre-enterprise, was quite an outcome.

The first theatre collective appeared in 1946 and was called "Austrian Yiddish Theatre". The ensemble was administratively and 
artistically directed by the well-known actor, H. Krelman. 
The theatre played the old Yiddish repetoire and
evoked recognition within the theatre-world.

The Jewish D.P camps in Austria got the possibility to see such theatre-shows as "The Dybbuk" (Ansky),
"God, man and the Devil" (Gordon), "Yankel Boyle" (Kobrin), "Khasia the Orphan" (Gordon) and "Hershele Ostropoler" (Gershenzon).

The collective did not have any special artistic ambitions, did not seek new ways or new plays; and, did not perform about the horrendous past which was then still
very recent. Rather they gave good and honest performances and that was greatly appreciated. The ensemble had their residence in Linz and was materially and morally
supported by the Joint\footnoteA{JDC (the American Jewish Joint Distribution Committee or "The Joint") is the leading Jewish humanitarian organization, working in 70 countries to lift lives and strengthen communities. We rescue Jews in danger, provide aid to vulnerable Jews, develop innovative solutions to Israel’s most complex social challenges, cultivate a Jewish future, and lead the Jewish community’s response to crises. For over 100 years, our work has put the timeless Jewish value of mutual responsibility into action, making JDC essential to the survival of millions of people and the advancement of Jewish life across the globe. \newline https://www.jdc.org} and the newly formed Jewish organisations.

\endnumbering
\end{english}
\end{Leftside}

\end{pairs}
\Columns


\end{document}




















































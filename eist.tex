\documentclass{article}
%\usepackage[margin=2in]{geometry}
\usepackage[osf,p]{libertinus}
\usepackage{microtype}
\usepackage[pdfusetitle,hidelinks]{hyperref}

\usepackage[series={A,B,C}]{reledmac}
\usepackage{reledpar}

\usepackage{graphicx}
\usepackage{polyglossia}
\setmainlanguage{english}
\setotherlanguage{hebrew}
\gappto\captionshebrew{\renewcommand\chaptername{קאַפּיטל}}
\usepackage{metalogo}


%%linenumincrement*{1}
%%\firstlinenum*{1}
%%\setlength{\Lcolwidth}{0.44\textwidth}
%%\setlength{\Rcolwidth}{0.44\textwidth}

\begin{document}
%%\maxhnotesA{0.8\textheight}
\renewcommand{\abstractname}{\vspace{-\baselineskip}}
\title{Fun noentu over, Vol I, p172.}
\author{Transl. Ilan Pillemer}
\date{\today}

\maketitle
\abstract{
Translation Exercise week 3
}
\newline

\begin{pairs}

\begin{Rightside}

\begin{RTL}
\begin{hebrew}
\beginnumbering
\autopar
\emph{
ייִדיש טעאַטער אין די „די־פּי“־לאַגערן פֿון עסטרײך (אַמעריקאַנד זאָנע)
}
\newline

אויך אין די עסטרײכישע ד“פּ־לאַגערן אין די יאָרן 1946־1947
געגרינדעט געװאָרן פּראָפֿעסיאָנעלע טעאַטער־גרופּן, װי אויך אַ צאָל דראַמאַטישע קרײזן.
 צוליב דער קלײנער צאָל געראַטעװעטע ײדין, װעלכע האָבן זיך געפֿונען אין עסטרײך, איז אויך די צאָל אָנטײלנעמער און
 דער פֿאַרנעם פֿון די טעאַטער־אונטערנעמונג געװען אַ באַשײדענער.

\endnumbering
\end{hebrew}
\end{RTL}
\end{Rightside}


\begin{Leftside}
\begin{english}
\beginnumbering
\autopar
\emph{
Yiddish theatre in the D.P camps in Austria (American Zone)
}
\newline 
 
Professional theatre-groups, as well as drama circles, were also founded in the Austrian D.P camps in the years 1946-1947.
On account of the small number of Jewish survivors who found themselves in Austria - both the number of participants; and, the forming of
the theatre-enterprise, was quite an outcome.

\endnumbering
\end{english}
\end{Leftside}

\end{pairs}
\Columns


\end{document}



















































